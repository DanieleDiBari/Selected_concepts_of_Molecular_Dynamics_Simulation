\documentclass[notes]{beamer}       		% compila sia i frame che le note
%\documentclass{beamer}              				% compila solo i frame
%\documentclass[notes=only]{beamer}  	% compila solo le note

% ita language and encoding
\usepackage[utf8]{inputenc}
%\usepackage[italian]{babel}

% formattazione
\usepackage{ragged2e} 							% pacchetto che contiene il comando \justifying
%\apptocmd{\frame}{}{\justifying}{}  	% --> applica la giustificazione a tutti i frame del documento
%\justifying											% --> applica la giustificazione a tutto il documento
\usepackage{hyperref}

% graphics style
\usepackage{graphicx}
\usepackage{xcolor}
\usepackage{shadowtext}

%tabelle
\usepackage{tabularx}
\usepackage{multirow}
\usepackage{booktabs} % aggiunge: \toprule , \midrule e \bottomrule ; stile tabella

%image paths
\graphicspath{	
	{./Images/}
}

% impostazioni base delle slide - tema, colori, caratteri, ... (deic.uab.es/~iblanes/beamer_gallery/)
\mode<presentation>
{
	%\usetheme{CambridgeUS}      	% or try Darmstadt, Madrid, Warsaw, ...
  	\usecolortheme{orchid} 				% for the color box (or try rose ...)
	\usefonttheme{structurebold} 	% or try structureitalicserif, structuresmallcapsserif
}



%% ---------------------------------------------------------------------------------------------------------
% %	CREAZIONE TEMA 

%%    Impostazione dei colori - COLORTHEME 
% (modifica di "beamercolorthemebeaver.sty" in "MiKTeX 2.9/tex/latex/beamer/themes/color" )
\definecolor{darkblue}{RGB}{52,57,176}

\setbeamercolor{section in toc}{fg=black,bg=white}
\setbeamercolor{alerted text}{fg=darkblue!80!gray}
\setbeamercolor*{palette primary}{fg=darkblue!60!black,bg=gray!30!white}
\setbeamercolor*{palette secondary}{fg=darkblue!70!black,bg=gray!15!white}
\setbeamercolor*{palette tertiary}{bg=darkblue!80!black,fg=gray!10!white}
\setbeamercolor*{palette quaternary}{fg=darkblue,bg=gray!5!white}

\setbeamercolor*{sidebar}{fg=darkblue,bg=gray!15!white}

\setbeamercolor*{palette sidebar primary}{fg=darkblue!10!black}
\setbeamercolor*{palette sidebar secondary}{fg=white}
\setbeamercolor*{palette sidebar tertiary}{fg=darkblue!50!black}
\setbeamercolor*{palette sidebar quaternary}{fg=gray!10!white}

%\setbeamercolor*{titlelike}{parent=palette primary}
\setbeamercolor{titlelike}{parent=palette primary,fg=darkblue}
\setbeamercolor{frametitle}{bg=gray!10!white}
\setbeamercolor{frametitle right}{bg=gray!60!white}

\setbeamercolor*{separation line}{}
\setbeamercolor*{fine separation line}{}


%%     Impostazione dei blocchi - INNERTHEME 
% (modifica di "beamerinnerthemerounded.sty" in "MiKTeX 2.9/tex/latex/beamer/themes/inner" )
%\setbeamertemplate{blocks}[rounded]
%\setbeamertemplate{items}[ball]
%\setbeamertemplate{sections/subsections in toc}[ball]
%\setbeamertemplate{title page}[default][colsep=-4bp,rounded=true]
%\setbeamertemplate{part page}[default][colsep=-4bp,rounded=true]


%	%    Impostazione delle intestazioni - OUTERTHEME
% (modifica di "beamerouterthemeinfolines.sty" in "MiKTeX 2.9/tex/latex/beamer/themes/outer" )
\setbeamercolor{institute in head/foot}{parent=palette secondary}
\setbeamercolor{subject in head/foot}{parent=palette tertiary}
\setbeamercolor{title in head/foot}{parent=palette primary}
\setbeamertemplate{footline}
{ 
	\leavevmode%
	\hbox{%
  		% "ht" è lo spazio sopra il carattere partendo dalla base del carattere, "dp"  è lo spazio sotto il carattere partendo sempre dalla base del carattere
  		\begin{beamercolorbox}[wd=.333333\paperwidth,ht=2.5ex,dp=1ex,left,leftskip=4ex]{author in head/foot}%
   			\usebeamerfont{author in head/foot}\insertshortauthor
  		\end{beamercolorbox}%
  		\begin{beamercolorbox}[wd=.333333\paperwidth,ht=2.5ex,dp=1ex,center]{institute in head/foot}%
    		\usebeamerfont{title in head/foot}\insertshortinstitute
 		 \end{beamercolorbox}%
 		 \begin{beamercolorbox}[wd=.333333\paperwidth,ht=2.5ex,dp=1ex,right,rightskip=4ex]{date in head/foot}%
 		 	\usebeamerfont{date in head/foot}\insertshortdate{}\hspace*{2em}\insertframenumber{} / \inserttotalframenumber
 		 \end{beamercolorbox}
 		 }%
 	\vskip0pt%
}
\makeatletter
\setbeamertemplate{headline}
{
  \leavevmode%
  \hbox{%
  \begin{beamercolorbox}[wd=.5\paperwidth,ht=2.8ex,dp=1.35ex,left]{subject in head/foot}	
    \usebeamerfont{section in head/foot}\hspace*{4ex}Complements of Condensed Matter Physics			% --> N.B. Qui c'è l'unica scritta che non usa nessun template tipo \title o \institute
  \end{beamercolorbox}%
  \begin{beamercolorbox}[wd=.5\paperwidth,ht=2.8ex,dp=1.35ex,right]{title in head/foot}%
    \usebeamerfont{subsection in head/foot}\insertshorttitle\hspace*{4ex}
  \end{beamercolorbox}}%
  \vskip0pt%
}
\makeatother

%%    FINE IMPOSTAZIONI TEMA
%% ---------------------------------------------------------------------------------------------------------



% settagio dati principali del file - titolo, autore, ...
\title[Selected concepts of Molecular Dynamics Simulation]{Molecular Dynamics Simulation}
\subtitle{A brief introduction of selected concepts}
\author{Daniele Di Bari}
\institute{University of Perugia}
\date{5th March 2018}
\subject{Complements of Condensed Matter Physics}

% By default the beamer class adds navigation buttons in the bottom right corner. To remove them one can place
\beamertemplatenavigationsymbolsempty

% settaggio prima slide
\setbeamertemplate{title page}
{
	\shadowcolor{white!30!black}
	\vspace{2.25cm}	   
	
	\hbox{
		\begin{beamercolorbox}[wd=.04\textwidth]{white}\end{beamercolorbox}%
  		\begin{beamercolorbox}
  		[wd=.92\textwidth,ht=.115\paperheight,dp=0.2cm,center,rounded=true]{subject in head/foot}%
   			\centering
   			\shadowtext{\textcolor{white}{\textbf{\LARGE \inserttitle}}}
   			
   			\vspace*{0.05cm}
			\shadowtext{\textcolor{white}{\emph{\large \insertsubtitle}}}		
  		\end{beamercolorbox}%
 	}
    	
%    \begin{colorbox}{darkblue}{ \parbox{0.8\paperwidth}{
%    	\shadowtext{\textcolor{white}{\textbf{\footnotesize{ }}}}
%    	
%    	\vspace{0.2cm}  	
%    	
%    	\shadowtext{\textcolor{white}{\textbf{\LARGE \inserttitle}}}\par
%    	
%    	\vspace{0.1cm}  
%    	
%    	\shadowtext{\textcolor{white}{\emph{\large \insertsubtitle}}}\par
%    	
%    	\vspace{0.2cm}  
%    	}}
%    \end{colorbox} 
%   	
   	\vspace{1.5cm}
	
	\titlepagetext{lightgray}{The general particle physics}\\
	\titlepagetext{lightgray}{experiment in space, on board}\\
	\titlepagetext{gray}{the International Space Station}\\
	\titlepagetext{gray}{since 19th May 2011.}		
}
				
% colors
\definecolor{itemred}{RGB}{163,0,0}
\definecolor{itemblue}{RGB}{52,57,176}
\definecolor{notegrey}{RGB}{90,90,90}
\definecolor{titlebgd_grey}{RGB}{242,242,242}
\definecolor{titleframe_red}{RGB}{204,0,0}

% COMANDI
% Formato testo generico in titolpage
\newcommand\titlepagetext[2]{\textcolor{#1}{\footnotesize{\textit{{#2}}}}}
\newcommand\colortextbf[2]{\textcolor{#1}{\textbf{#2}}}
\newcommand\bluetextbf[1]{\textcolor{itemblue}{\textbf{#1}}}
\newcommand\bluetextit[1]{\textcolor{itemblue}{\textit{#1}}}
\newcommand\textitem[2]{\textcolor{itemblue}{\textbf{#1} (\textit{#2})}}

% Testo giustificato
\newenvironment<>{justify}{\justifying{}}

\begin{document}
	% impostare una foto come sfondo del frame: 
	%  \usebackgroundtemplate{\includegraphics[height=\paperheight,keepaspectratio]{on_ISS-08_cutMIDDLEUP-3088.jpg}}	
	% 	possibili impostazioni per l'img: width=\paperwidth  --  height=\paperheight  --  keepaspectratio

\begin{frame}\pdfbookmark[2]{Title}{Title}
	\titlepage
\end{frame}

\usebackgroundtemplate{}

\begin{frame}[allowframebreaks]\pdfbookmark[2]{Contents}{Contents}
\frametitle{Table of Contents}
\tableofcontents
\end{frame}

\section{Introduction}
\begin{frame}<presentation:0>[noframenumbering]
	\setbeamertemplate{background}{white}
	\frametitle{Introduction}
	\framesubtitle{The problem of describing complex system}
	\justifying
	
	Few systems in statistical mechanics are exactly soluble, the majority require approximations, but when their complexity increases 
	it becomes difficult even to construct an approximate theory in a reasonable way. 	
	
	\vspace*{0.25cm}
	
	\hspace*{0.3cm}
	\begin{beamercolorbox}[wd=.92\textwidth,center,rounded=true]{institute in head/foot}
   			The more complex and realistic is a system,\\
   			the more difficult and interesting it becomes to describe it.
  	\end{beamercolorbox}%
  	
  	\vspace*{0.2cm}
		
	In this case, an increase in the complexity raises the necessity of having exact results available, both to test existing 
	approximation methods and to point the way towards new approaches. 	
	
	\vspace*{0.2cm}
	
%	\begin{itemize}\justifying
%		\item	\textbf{Computer simulations allow to calculate the essentially exact results for a given model without having to rely on 
%				approximate theories.} 
%	\end{itemize}
	
	\begin{columns}
		\begin{column}{0.1\textwidth}
		\raggedleft
			\hspace*{0.5cm}\textcolor{itemblue}{$\blacktriangleright$}
			
			\vspace*{0.2cm}
			
			\hspace*{0.5cm}\textcolor{itemblue}{$\blacktriangleright$}
		\end{column}
		
		\begin{column}{0.8\textwidth}\justifying
			\textbf{Computer simulations allow to calculate the essentially exact results for a given model without having to rely on 
				approximate theories.} 
		\end{column}
		
		\begin{column}{0.1\textwidth}
			\textcolor{itemblue}{$\blacktriangleleft$}
			
			\vspace*{0.2cm}			
			
			\textcolor{itemblue}{$\blacktriangleleft$}
		\end{column}
	\end{columns}
\end{frame}

\begin{frame}[label=intro-complex_sys]
	\frametitle{Introduction}
	\framesubtitle{The problem of describing complex system}
	\justifying
	
	Few systems in statistical mechanics are exactly soluble, the majority require approximations, but when their complexity increases 
	it becomes difficult even to construct an approximate theory in a reasonable way. 	
	
	\vspace*{0.4cm}
	
	\hspace*{0.3cm}
	\begin{beamercolorbox}[wd=.92\textwidth,center,rounded=true]{institute in head/foot}
   			The more complex and realistic is a system,\\
   			the more difficult and interesting it becomes to describe it.
  	\end{beamercolorbox}%
  	
  	\vspace*{0.35cm}
		
	In this case, an increase in the complexity raises the necessity of having exact results available, both to test existing 
	approximation methods and to point the way towards new approaches. 	
		
%	\begin{itemize}\justifying
%		\item	\textbf{Computer simulations allow to calculate the essentially exact results for a given model without having to rely on 
%				approximate theories.} 
%	\end{itemize}

\end{frame}


\begin{frame}[label=intro-why_use_CS]
	\frametitle{Introduction}
	\framesubtitle{Why to use computer simulations?}
	\justifying
	
	Computer simulations allow to calculate the essentially exact results for a given model without having to rely on approximate theories.
	
	\begin{itemize}\small{\justifying
		\item[$\blacktriangleright$] 	\textbf{TEST THE MODELS:} comparing the results of the simulation with those of a real experiment. 
												If there is disagreement, the model is inadequate - it is necessary to improve on the estimate of the 
												intermolecular interactions.

		\vspace*{0.1cm}												

		\item[$\blacktriangleright$] 	\textbf{TEST THE THEORIES:} comparing the results of the simulation with the predictions of an 
												approximate analytical theory applied to the same model. If there is disagreement, the theory is 
												flawed - the simulation plays the role of the experiment designed to test the theory. 
												This method of screening theories before we apply them to the real world is called a 
												\textit{computer experiment}.

		\vspace*{0.1cm}												
												
		\item[$\blacktriangleright$] 	\textbf{MAKE NUMERICAL PREDICTIONS:} useful to describe a tho- ught experiment and to 
												integrate the macroscopic information with the atomistic level provided by the simulation.
	}\end{itemize}	
	
	%\footnotetext[1]{\justifying The computer simulation plays the role of the experiment designed to test the theory. This method of 
	%screening theories before we apply them to the real world is called a \textit{computer experiment}.}
\end{frame}


\begin{frame}[label=intro-virtual_lab]
	\frametitle{Introduction}
	\framesubtitle{Computer simulations as virtual laboratories}
	\justifying
	
	Computer simulations are similar to a \textbf{virtual laboratory} and, as a consequence, they have several advantages:
	
	\begin{enumerate}\justifying
		\item	Many \textit{virtual experiments can be easily set up} and carried out in succession by simply varying the control 
				parameters
		\item \textit{Extreme\textcolor{white}{y}conditions}, such as high $T$ and $p$, 
				\textit{can\textcolor{white}{y}be created in a simple and considerably safety manner}
		\item	The \textit{time-scales of a system present no impediment} to the simulator thus a wide range of  physical phenomena, 
				from molecular to the galactic, may be studied using computer simulation. 
												%There is no difference between $10^{15} s$ or $10^{-9} s$
	\end{enumerate}

	Technologically useful informations, for systems that may be difficult or even impossible to investigate directly with a real 
	experiment (e.g. nuclear reactors or planetary cores), can be obtained from those type of simulations. 

\end{frame}


\section{Molecular simulations}
\begin{frame}[label=molecular_sim]
	\frametitle{Molecular simulations}
	\framesubtitle{From computer simulations to molecular simulations}
	\justifying
	
	Molecular simulations are computer simulations that study systems from a molecular point of view. 
	\textbf{Starting from an atomistic level, this simulations are used to 
	predict and better understand the properties of complex materials}.\textbf{\footnotemark[1]}
	
	\vspace*{0.25cm}
	
	Molecular simulations provide \underline{a direct route from the microscopic} \underline{details of a system}
	(the masses of the atoms, the	interactions between them, etc.) \underline{to macroscopic properties of experimental interest}
	(the equation of state, transport coefficients and so on).
	
	\vspace*{0.25cm}

	The usefulness of this type of simulations rely on the fact that \textbf{a sample containing a few thousand of particles usually is sufficiently large to simulate the behaviour of a macroscopic system}.
	
	\footnotetext[1]{Even the materials that have not yet been made.}
\end{frame}

%\begin{frame}[label=intro-molecular_sim2]
%	\frametitle{Introduction}
%	\framesubtitle{Molecular simulations - II}
%	\justifying
%	
%	It is clear that molecular simulations have to be able to computing at least the equilibrium properties of a 
%	\textbf{many-body system} and, hopefully, also those of transporting process.
%	
%	\vspace*{0.25cm}
%
%	To perform this type of numerical simulation, it is necessary to \textbf{sample the phase-space} of the system. 
%	When this is done, one has the positions and momenta of all the particles of the system. Thus it is possible to the evaluate the
%	microscopic value of all observables and, eventually, to average them over time or phase-space.
%
%	\vspace*{0.25cm}
%	
%	The downside is that the results of this simulations are only as good as the numerical model and the results can be artificially 
%	biased if the simulation is unable to sample an adequate number of microstates over the time it is allowed to run.
%\end{frame}

\begin{frame}[label=molecular_sim-stat_mech]
	\frametitle{Molecular simulations}
	\framesubtitle{Molecular simulations and statistical mechanics}
	\justifying

	Molecular simulations allow the \textbf{study of the properties of many-particle systems}. 
	However, not all properties can be directly measured in a simulation. Conversely, most of the quantities that can be measured in a 
	simulation do not correspond to properties that are measured in real experiments.
	
	\vspace*{0.25cm}
	
	Actually, \underline{molecular simulations generate information at the micro}-\\
	\underline{scopic level} (atomic and molecular positions, velocities, etc.) 
	and the conversion of this very detailed information into macroscopic terms (pressure, internal energy, etc.) is the field of the 
	statistical mechanics. 
	Thus, the language of \textbf{statistical mechanics is necessary to use these simulations as the numerical counterpart of experiments}.
	
\end{frame}


\subsection{Principal steps}
\begin{frame}[label=molecular_sim-principal_steps]
	\frametitle{Molecular simulations}
	\framesubtitle{The principal steps of molecular simulations}
	\justifying
	
	\vspace*{0.1cm}

	A molecular-scale computer simulation consists of 3 principal steps:
	
	\begin{enumerate}\justifying
		\item Construction of a model
		\item Calculation of molecular trajectories
		\item Analysis of those trajectories to obtain property values
	\end{enumerate}
	
	\vspace*{0.2cm}

	The second step constitutes the \textbf{main feature} of the simulation. In a way that molecular positions $\bold{r}^N$ are computed 
	in this step, thus it is possible to discriminate among the different simulation methods.
	
	\vspace*{0.2cm}
	
	\begin{center}
		\includegraphics[width=0.7\textwidth]{different_tipes_of_CS.PNG}
	\end{center}
	
\end{frame}


\subsection{Different types}
\begin{frame}[label=molecular_sim-diff_typesd]
	\frametitle{Molecular simulations}
	\framesubtitle{Different types of molecular simulations}
	\justifying
	
	\textbf{Molecular Dynamics -} The positions are obtained by numerically solving differential equations of motion and hence they 
	are connected in time - the positions reveal dynamics of individual molecules as in a motion picture. 
	
	\vspace*{0.3cm}
	
	\textbf{Monte Carlo -} The molecular positions are not temporally related, but are generated stochastically such that a molecular 
	configuration $\bold{r}^N$ depends only on the previous configuration.\footnotemark[1]
	
	\vspace*{0.3cm}
	
	\textbf{Others simulation methods - } The positions are computed from hybrid schemes that use some stochastic features, as 
	in Monte Carlo, and some deterministic features, like in Molecular Dynamics.
	
	\footnotetext[1]{\justifying When the outcome of a random event in a sequence depends only on the outcome of the immediately 
	previous one, the sequence is called: \textit{Markov chain}.}
\end{frame}


\subsubsection{Molecular Dynamics or  Monte Carlo}
\begin{frame}[label=molecular_sim-MCSvsMDS1]
	\frametitle{Molecular simulations}
	\framesubtitle{Molecular Dynamics or  Monte Carlo - I}
	\justifying
	
	The choice between Molecular Dynamics (MD) and Monte Carlo (MC) is largely determined by the phenomenon under investigation.
	
	\begin{itemize}\justifying\small{
		\item \textbf{Monte Carlo is preferable for GAS (}\textit{or other low density systems}\textbf{):}\\
		There can be large energy barriers (of several $k_B T$) which can lead to molecules being trapped in a few low energy 
		conformations in a MD simulation, leading to poor conformational sampling.
		In contrast, the random moves in a MC simulation can easily lead to barrier crossings. 
		
		\vspace*{0.07cm}
		
		\item \textbf{Molecular Dynamics is preferable for LIQUIDS:}\\
		Molecular Collisions exchange energy between them, enabling barrier crossings, improving the ability of MD to sample conformations.
		For a MC simulation, there is a large probability of selecting random moves for which two or more molecules overlap leading to large 
		number of rejected moves and a decrease in efficiency of sampling.\footnote{However, the ability of MC to make unphysical moves, for example to flip a molecule around, can in some cases compensate for this.}
	}\end{itemize}
	
%[Dr S J Clark - http://cmt.dur.ac.uk/sjc/thesis_dlc/node58.html]
\end{frame}


\begin{frame}[label=molecular_sim-MCSvsMDS2]
	\frametitle{Molecular simulations}
	\framesubtitle{Molecular Dynamics or  Monte Carlo - II}
	\justifying
			
	In any cases, there are some situations where only one method is appropriate. In particular:
	
	\vspace*{0.2cm}

	\begin{enumerate}\justifying
		\item	\textbf{Determination of transport properties}, such as viscosity coefficients, is largely only possible using
					\textbf{Molecular Dynamics}, as Monte Carlo lacks an objective definition of time 
					(except in some special cases such as the bond fluctuation model for polymers). 
					
		\vspace*{0.2cm}

		\item	%On the other hand, \textbf{Monte Carlo} can be used for simulations with \textbf{varying particle numbers} 
					%(Grand Canonical Monte Carlo) by adding moves for the creation and destruction of particles.
					Simulations with \textbf{varying particle numbers} that can be made using \textbf{Monte Carlo} methods
					(Grand Canonical Monte Carlo) by adding moves for the creation and destruction of particles.
	\end{enumerate}	 
	
%[Dr S J Clark - http://cmt.dur.ac.uk/sjc/thesis_dlc/node58.html]
\end{frame}
	

\section{Monte Carlo simulation}
\begin{frame}[label=MCS-intro1]
	\frametitle{Monte Carlo simulations}
	\framesubtitle{A summary of Monte Carlo simulations - I}
	\justifying
	
	Monte Carlo simulation is a purely stochastic method typically used to evaluate statistical-mechanical ensemble averages, i.e. 
	multidimensional integrals performed of a $N$-body system. 
	
	\vspace*{0.3cm}
	
	Those integrals are evaluated by accumulating the integrand at random generated values of the independent variables that define 
	a configuration of the system (usually the position $ \bold{r}^N$, possibly also the volume of the system or the number of particles 
	that it contains). 
	
	\vspace*{0.3cm}
	
	\textbf{The particle momenta do not enter the calculation, there is no time scale involved, and the order in which the 
	configurations occur has no special significance. The method is therefore limited to the calculation of static properties.}
	
\end{frame}


\begin{frame}[label=MCS-intro2]
	\frametitle{Monte Carlo simulations}
	\framesubtitle{A summary of Monte Carlo simulations - II}
	\justifying
	
	\vspace*{-0.2cm}
	\textbf{Not all configurations that are generated are accepted.} 
	The choice of rejecting or not a new configuration is made in such a way that, asymptotically, the configuration space is sampled 
	according to the probability density corresponding to a particular ensemble.\footnotemark[1] 
	
	For example, in the canonical ensemble the average of a microscopic variable $A ( \bold{r}^N )$ is:
	\begin{equation}\label{eq:can_ens:avg_val}
		\left< A \right> = \frac{1}{Z}	\int \cdots \int \, A ( \bold{r}^N ) \; e^{ -\beta \, U( \bold{r}^N)} \; d\bold{r}_1 \cdots d\bold{r}_N
	\end{equation}
	Due to $e^{ -\beta \, U( \bold{r}^N)}$, some configurations contribute largely to the integral while others do it less. 
	Thus, \textbf{it is necessary to bias the sampling in favour of those configurations most likely to occur}.
	
	\footnotetext[1]{\justifying In general, configurations that go against at least one of the system's constrains,  
	e.g. due to an overlap of two hard-spheres, have to be rejected.}

\end{frame}


\subsection{Metropolis method}
\begin{frame}[label=MCS-Metropolis1]
	\frametitle{Monte Carlo simulations}
	\framesubtitle{A summary of the Monte Carlo simulations: Metropolis method - I}
	\justifying	
	
	\vspace*{0.2cm}
	
	The Metropolis method used in Monte Carlo simulations involves the following main steps:
	
	\begin{columns}
		\begin{column}{0.00\textwidth}\end{column}
		
		\begin{column}{0.68\textwidth}\justifying
			\begin{enumerate}\justifying
				\item	\textbf{Starting from a configuration} $m$. If this is the 1st step, initializing the positions and computing the 
				potential energy.
				\item \textbf{Generating a trial configuration} $n\,$ by selecting a particle $i\,$ at random and giving it a small, random
						displacement, $\bold{r}^n_i \rightarrow \bold{r}^n_i + \Delta\bold{r}$, where $\Delta\bold{r}$ is chosen uniformly 
						within prescribed limits.
			\end{enumerate}
		\end{column}
		
		\begin{column}{0.32\textwidth}
			\centering	
			\includegraphics[width=\textwidth]{Metropolis.PNG}
		\end{column}	
		
		\begin{column}{0.00\textwidth}\end{column}
	\end{columns}	
	
	\begin{columns}
		\begin{column}{0.705\textwidth}\end{column}
		
		\begin{column}{0.35\textwidth}
			\centering	
			
			\vspace*{-0.3cm}
			\scriptsize{\bluetextbf{	Schematic view of
											a random displacement
											of the $i$-particle from $\bold{r}^m_i$ to $\bold{r}^n_i$}}					
		\end{column}	
		
		\begin{column}{0.00\textwidth}\end{column}
	\end{columns}	
	
\end{frame}


\begin{frame}[label=MCS-Metropolis2]
	\frametitle{Monte Carlo simulations}
	\framesubtitle{A summary of the Monte Carlo simulations: Metropolis method - II}
	\justifying
	
	\begin{enumerate}\justifying
		\item[3.] \textbf{Verifying} whether \textbf{the trial configuration} must be rejected:
					\begin{itemize}\justifying
						\item \bluetextbf{yes:} count the old configuration $n$ as the new configuration $m$ and repeat the process from 
								step 1
						\item \bluetextbf{not:} compute the new potential energy $U_n$ and accept the proposed configuration with the 
								following:
								
								\vspace*{-0.5cm}
								\begin{equation*}
									\hspace*{0.35cm}\begin{cases}
									\text{if } U_n < U_m \Longrightarrow \text{accept with probability} = 1\\
									\text{if } U_n > U_m \Longrightarrow \text{accept with probability} \propto  e^{ -\beta \, (U_n\, - U_m)} 
									\end{cases}	
								\end{equation*} 
					\end{itemize}
		\item[4.] \textbf{Evaluating property integrands}, such as $A(\bold{r}^N)$ in \eqref{eq:can_ens:avg_val}, and accumulating them in 
					running sums
		\item[5.] \textbf{Iterate the entire process} until at least a few million of configurations have been made to obtain an adequate 
					statistical error in the average values
	\end{enumerate}
	
\end{frame}


\section{Molecular Dynamics Simulations}
\begin{frame}[label=MDS-intro1]
	\frametitle{Molecular Dynamics Simulations}
	\framesubtitle{Computing the internal motion of classical many-body systems}
	\justifying

	Molecular Dynamics simulations compute the motions of individual molecules for a classical many-body system in order to describe the 
	equilibrium and transport properties of solids, liquids and gasses.	
	
	\vspace*{0.2cm}
	
	\hspace*{0.25cm}
	\begin{beamercolorbox}[wd=.93\textwidth,center,rounded=true]{institute in head/foot}
	\small{
   			Although this modelling of the matter at the microscopic level
   			%(\textit{based on a comprehensive description of the constituent particle})\\
   			must be, in principle, based on quantum mechanics,\\
   			\hspace*{0.1cm}\textbf{Molecular Dynamics generally adopts a classical point of view}
   	}
  	\end{beamercolorbox}%
	
	\vspace*{0.2cm}
	
	In this context, the word classical means that the nuclear motion of the constituent particles obeys the laws of classical mechanics 
	(the motion is described by the second Newton's law). 
	
	\vspace*{0.2cm}
		
	This is an \textbf{excellent approximation for a wide range of materials}.	
\end{frame}


\begin{frame}[label=MDS-intro2]
	\frametitle{Molecular Dynamics Simulations}
	\framesubtitle{Classical approximation}
	\justifying
	
	\vspace*{-0.25cm}
	
	\begin{block}{\small{Neither relativistic nor quantum effects are considered}}\justifying\small{
		\textbf{Special relativity} does not allow information to travel faster than light; Molecular Dynamics simulations assume forces with 
		an infinite speed of propagation. 
	
		\textbf{Quantum mechanics} has at its base the uncertainty principle; Molecular Dynamics requires, and provides, complete 
		information about position and momentum at all times.
	}\end{block}

	\vspace*{0.05cm}
	
	In practice, the phenomena studied by Molecular Dynamics simulation are those where relativistic effects are not observed and 
	quantum effects can, if necessary, be incorporated as semi-classical corrections derived from quantum theory.\footnotemark[1]

	\footnotetext[1]{\justifying For example, dealing with very light atoms or molecules (e.g. He, $\text{H}_2$, $\text{D}_2$) or with 
	vibrational motions whose characteristic frequency, translated in energy, is comparable or larger than $k_B T$, quantum effects became 
	not negligible.}
\end{frame}

\begin{frame}
	\frametitle{Introduction}
	\framesubtitle{What is a Molecular Dynamics Simulation?}
	\justifying

	
%	One of the main approximation usually done in molecular simulation is to assume that the motion of the nuclei can be described by 
%	Newton's law. Thus it is possible to use the theoretical background of the classical statistical mechanics to measure the thermodynamic 
%	averages. 
%
%	This is an excellent approximation for a wide range of materials.
%	Only when we consider the translational or rotational motion of light atoms or molecules (He, $\text{H}_2$, $\text{D}_2$) 
%	or vibrational motion with a frequency $\nu$ such that $h \nu < k_B T$ should we worry about quantum effects.
\end{frame}

\begin{frame}
	\frametitle{Introduction}
	\framesubtitle{What is a MD Simulation?}
	\justifying
	Molecular dynamics simulations compute the motions of individual molecules in models of solids, liquids and gasses. The key idea is motion, which describes how positions, velocities and orientations change with time. In effect, molecular dynamics constitutes a motion picture that follows molecules as they dart to and fro, twisting, turning, colliding with one another and, perhaps, colliding with their container.
\end{frame}

\begin{frame}
	\frametitle{Introduction}
	\framesubtitle{What is a MD Simulation?}
	\justifying
Historically, the choice
of algorithm was often determined in large part by the amount of computer
memory needed, i.e. the number of variables that needed to be kept track of.
Given the large memories available today, this concern has been largely
ameliorated. Two features that we do want to mention here which were
introduced to make molecular dynamics simulations faster are potential ``cutoffs''
and ``neighbor lists''. (These labor saving devices can also be used for
Monte Carlo simulations of systems with continuous symmetry.) As the
particles move, the forces acting on them change and need to be continuously
recomputed. A way to speed up the calculation with only a modest
reduction in accuracy is to cut off the interaction at some suitable range and
then make a list of all neighbors which are within some slightly larger radius.
As time progresses, only the forces caused by neighbors within the ``cutoff
radius’ need to be recomputed, and for large systems the reduction in effort
can be substantial. (The list includes neighbors which are initially beyond
the cutoff but which are near enough that they might enter the ``interacting
region'' within the number of time steps, typically 10–20 which elapse before
the list is updated.) With the advent of parallel computers, molecular
dynamics algorithms have been devised that will distribute the system
over multiple processors and allow treatment of quite large numbers of
particles. One major constraint which remains is the limitation in maximum
integration time and algorithmic improvement in this area is an important
challenge for the future.
\end{frame}

\section{Technical Requirements}
\subsection{from Physics Goals}
\begin{frame}
    \frametitle{Technical Requirements}
    \framesubtitle{from Physics Goals}
    \justifying
    
	Physical goals involve \textbf{technical requirements for the AMS detector}:
	
	\begin{itemize}\justifying
		\item	\bluetextbf{Dark Matter $\Rightarrow$}
					to get $e^+/p$ rejection of $\sim10^{-6}$ for the measurement of the positron fraction.
		\item	\bluetextbf{Matter/Antimatter Asymmetry $\Rightarrow$}
					to reach a sensitivity in the search for anti-matter nuclei of $10^{-10}$ (ratio of anti-helium nuclei to helium nuclei).
		\item	\bluetextbf{Cosmic Ray Physics $\Rightarrow$}
					to measure the composition and spectra of charged particles with an accuracy of 1\%.
	\end{itemize}
	
	Moreover, \bluetextbf{for each of these ones}, it is very important to extend, as far as possible, the energy range of the 
	measurements.
	
	For AMS, the required energy range is from 0.5 to $\sim$ 2000 $GeV$.

	%\footnotesize{\underline{\textbf{Note}} These requirements represent a considerable sensitivity improvement compared to the previous space-borne experiments.}

\end{frame}

\note{
	In a conventional molecular dynamics simulation of a bulk fluid a system of N particles is allocated a set of initial coordinates within a cell of fixed volume, most commonly a cube. A set of velocities is also assigned, usually drawn from a Maxwell distribution appropriate to the temperature of interest and selected in such a way that the net linear momentum of the system is zero. The subsequent calculation tracks the motion of the particles through space by integration of the classical equations of motion. Equilibrium properties are obtained as time averages over the dynamical history of the system in the manner outlined in Section 2.2 and correspond to averages over a microcanonical ensemble. In modern work N is typically of order 103 or 104, though much larger systems have occasionally been studied. To minimise surface effects, and thereby simulate more closely the behaviour expected of a macroscopic system, it is customary to use a periodic boundary condition.
}

\begin{frame}
	\frametitle{Technical Requirements}
    \framesubtitle{from Physics Goals}	
    
    \begin{block}{The technical challenge of AMS-02}
		\justifying
		The illustrated requirements represent a considerable sensitivity improvement compared to the previous space-borne 
		experiments.
	\end{block}
   	
   	\vspace{0.25cm}
	\small{\bluetextbf{NOTE}}
	\vspace{-0.2cm}
    \begin{itemize}
    	\item[\footnotesize{$\blacktriangleright$}] 
    		\footnotesize{\justifying
    		There is a strong demand for precision measurements of CRs in the energy region from 10 to 1000 
    		$GeV$ as the measurements of positron fraction, i.e.  $e^+ / (e^+ + e^-)$, by AMS-01, HEAT, PAMELA and FERMI 
    		indicate a large deviation of this ratio from the production of $e^+$ and $e^-$ predicted by a model that includes only 
    		ordinary CR collisions.
    		
    		These available measurements are both at too low an energy and of too limited statistics to shed the light on the origin of 
    		this significant excess.
    		
    		AMS-02 is expected to provide definitive answers concerning the nature of this deviation.}
   	\end{itemize}  
   	 
%
%%	\vspace{0.25cm}
%	There is a strong demand for precision measurements of cosmic rays in the energy region from 10 to 1000 GeV as the 
%	measurements of e+/(e+ + e-) by AMS-01, HEAT, PAMELA and FERMI indicate a large deviation of this ratio from the 
%	production of e+ and e- predicted by a model that includes only ordinary cosmic ray collisions. 
%	
%	These available measurements are both at too low an energy and of too limited statistics to shed the light on the origin of this 
%	significant excess. 
%	
%	AMS-02 is	expected to provide definitive answers concerning the nature of this deviation.
\end{frame}

%\begin{frame}
%	\frametitle{Technical Requirements}
%    \framesubtitle{From Physics Goals}
%	\begin{columns}
%		\begin{column}{0.5\textwidth}
%			\begin{center}
%				\includegraphics[width=0.95\columnwidth]{Schem_Requirements-DM.png}
%			\end{center}
%		\end{column}
%
%		\begin{column}{0.5\textwidth}
%			\begin{center}
%				\includegraphics[width=0.95\columnwidth]{Schem_Requirements-Antimatter.png}
%			\end{center}
%		\end{column}
%	\end{columns}
%\end{frame}

\begin{frame}[label=AMS-apparatus_img]
	\frametitle{Technical Requirement}
	\framesubtitle{general structure of the AMS apparatus}
	\begin{center}
		\vspace{-0.15cm}
		%\hspace{-0.35cm}\includegraphics[height=0.7\paperheight]{Schem_structure_of_AMS.png}

		\vspace{-0.35cm}

		\hspace{0.25cm}\hyperlink{AMS-apparatus_list}{\textbf{\beamerbutton{link to AMS apparatus list}}}
	\end{center}
	%\vspace{-0.25cm}
\end{frame}

\section{Bibliography}
\begin{frame}[allowframebreaks]
	\frametitle{Sources}
	\justifying
		
	\bluetextbf{Books on Molecular Dynamics Simulation}
	
	\begin{thebibliography}{99}
	\setbeamertemplate{bibliography item}[text]
	\scriptsize{
	\bibitem{books:Allen}
	Allen, M., \& Tildesley, D. (1987). \textit{Computer Simulation of Liquids}. Oxford, England, Clarendon Press.

	\bibitem{books:Frenkel}
	Frenkel, D., \& Smit, B. (2012). \textit{Understanding molecular simulation: from algorithms to applications}. 
	San Diego, Calif.: Academic Press.	
	
	\bibitem{books:Haile}
	Haile, J. M. (1992). \textit{Molecular dynamics simulation: elementary methods}. New York u.a.: Wiley.
	
	\bibitem{books:Hansen}
	Hansen, J. P., \& McDonald, I. R. (2014). \textit{Theory of simple liquids: with applications to soft matter}. Amsterdam: Elsevier. 
	
	\bibitem{books:Tuckerman}
	Tuckerman, M. E. (2010). \textit{Statistical Mechanics: Theory and Molecular Simulation}. New York, United States: 
	Oxford University Press. 
	
	
	\bibitem{site:Cheung}
	Cheung, D. L. (2002). \textit{Structures and Properties of Liquid Crystals and Related Molecules from Computer Simulation} 
	 (Ph.D. Thesis).\\Retrieved from \url{http://cmt.dur.ac.uk/sjc/thesis_dlc/}
	 
	 }

	\framebreak
	\hspace{-0.7cm}\normalsize{\bluetextbf{On the Transition Radiation Detector (TRD):}}		
	
	\scriptsize{
	\bibitem{Kirn-1}
	T. Kirn, ``\textit{The AMS-02 transition radiation detector}'' Nuclear Instruments and Methods in Physics Research A 581 
	(2007) 156-159.
	
	\bibitem{Kirn-2}
	T. Kirn, ``\textit{The AMS-02 TRD on the international space station}'' Nuclear Instruments and Methods in Physics Research A 
	706 (2013) 43-47.
	
	\bibitem{Sun}
	W. Sun, A. Kounine and Z. Weng, ``\textit{Measurement of the absolute charge of cosmic ray nuclei with the AMS Transition 
	Radiation Detector}'' 33rd International Cosmic Ray Conference, Rio de Janeiro (2013).
	}
	
	\vspace{0.25cm}
	\hspace{-0.7cm}\normalsize{\bluetextbf{On the Time of Flight detector (ToF):}}
	
	\scriptsize{
	\bibitem{Bindi-1}
	V. Bindi et al., ``\textit{The scintillator detector for the fast trigger and time-of-flight (TOF) measurement of the space 
	experiment AMS-02}'' Nuclear Instruments and Methods in Physics Research A 623 (2010) 968-981.
	
	\bibitem{Bindi-2}
	V. Bindi et al., ``\textit{The AMS-02 time of flight (TOF) system: construction and overall performances in space}'' 33rd 
	International Cosmic Ray Conference, Rio de Janeiro (2013).
	
	\bibitem{Bindi-3}
	V. Bindi et al., ``\textit{Calibration and performance of the AMS-02 time of flight detector in space}'' Nuclear Instruments and 
	Methods in Physics Research A 743 (2014) 22-29.
	}
	
	\framebreak
	\hspace{-0.7cm}\normalsize{\bluetextbf{On the Magnetic Spectrometer}}		
	
	\vspace{0.25cm}
	\hspace{-0.7cm}\small{\bluetextbf{$\triangleright$ Magnet:}}
	
	\scriptsize{
	\bibitem{Ahlen}
	S. Ahlen et al., ``\textit{An antimatter spectrometer in space}'' Nuclear Instruments and Methods in Physics Research A 350 
	(1994) 351-367.
	
	\bibitem{Lubelsmeyer}
	K. L\"{u}belsmeyer et al., ``\textit{Upgrade of the Alpha Magnetic Spectrometer (AMS-02) for long term operation on the 
	International Space Station (ISS)}'' Nuclear Instruments and Methods in Physics Research A 654 (2011) 639-648.
	}
	
	\vspace{0.25cm}
	\hspace{-0.7cm}\small{\bluetextbf{$\triangleright$ Silicon Tracker:}}
	
	\scriptsize{
	\bibitem{Alcaraz}
	J. Alcaraz et al., ``\textit{The alpha magnetic spectrometer silicon tracker: Performance results with protons and helium
	nuclei}'' Nuclear Instruments and Methods in Physics Research A 593 (2008) 376-398.

	\bibitem{Ambrosi-01}
	G. Ambrosi et al., ``\textit{AMS-02 Track reconstruction and rigidity measurement}'' 33rd International Cosmic Ray
	Conference, Rio de Janeiro (2013).
	
	\bibitem{Ambrosi-02}
	G. Ambrosi et al., ``\textit{Nuclear Charge Measurement With the AMS-02 Silicon Tracker}'' 33rd International Cosmic Ray
	Conference, Rio de Janeiro (2013).

	\bibitem{Ambrosi-03}
	G. Ambrosi et al., ``\textit{In-flight operations and status of the AMS-02 silicon tracker}'' PoS (ICRC2015) 690.
	
	\bibitem{Ambrosi-04}
	G. Ambrosi et al., ``\textit{Nuclei Charge measurement with the AMS-02 Silicon Tracker}'' PoS (ICRC2015) 429.	

	\bibitem{Azzarello}
	P. Azzarello, ``\textit{Tests And Production Of The AMS-02 Silicon Tracker Detectors}'' PhD thesis, University of Geneva 
	(2004).
	
	\bibitem{Batignani}
	G. Batignani et al., ``\textit{DOUBLE-SIDED READOUT SILICON STRIP DETECTORS FOR THE ALEPH MINIVERTEX}'' Nuclear 
	Instruments and Methods in Physics Research A 277 (1989) 147-153.

	\bibitem{Haas}
	D. Haas, ``\textit{The Silicon Tracker of AMS02}'' Nuclear Instruments and Methods in Physics Research A 530 (2004) 
	173-177.
	
	\bibitem{Krammer}
	M. Krammer, ``\textit{Silicon Detectors}'' XI ICFA School on Instrumentation (2010).
	
	\bibitem{Lutz}
	G. Lutz, ``\textit{Semiconductor Radiation Detectors}'' Springer 1st ed. (1999).
	
	\bibitem{Peisert}
	A. Peisert, ``\textit{Silicon microstrip detectors}'' Instrumentation in High Energy Physics (1992) 1-79.
	
	\bibitem{Pohl}
	M. Pohl, ``\textit{AMS tracking in-orbit performance}'' PoS (VERTEX2015) 022
	}
	
	\end{thebibliography}
\end{frame}



\end{document}
