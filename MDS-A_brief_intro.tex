\documentclass[notes]{beamer}       		% compila sia i frame che le note
%\documentclass{beamer}              				% compila solo i frame
%\documentclass[notes=only]{beamer}  	% compila solo le note

% ita language and encoding
\usepackage[utf8]{inputenc}
%\usepackage[italian]{babel}

% formattazione
\usepackage{ragged2e} 							% pacchetto che contiene il comando \justifying
%\apptocmd{\frame}{}{\justifying}{}  	% --> applica la giustificazione a tutti i frame del documento
%\justifying											% --> applica la giustificazione a tutto il documento

% graphics style
\usepackage{graphicx}
\usepackage{xcolor}
\usepackage{shadowtext}

%tabelle
\usepackage{tabularx}
\usepackage{multirow}
\usepackage{booktabs} % aggiunge: \toprule , \midrule e \bottomrule ; stile tabella

%image paths
\graphicspath{	
	{./Images/}
}

% impostazioni base delle slide - tema, colori, caratteri, ... (deic.uab.es/~iblanes/beamer_gallery/)
\mode<presentation>
{
	%\usetheme{CambridgeUS}      	% or try Darmstadt, Madrid, Warsaw, ...
  	\usecolortheme{orchid} 				% for the color box (or try rose ...)
	\usefonttheme{structurebold} 	% or try structureitalicserif, structuresmallcapsserif
}



%% ---------------------------------------------------------------------------------------------------------
% %	CREAZIONE TEMA 

%%    Impostazione dei colori - COLORTHEME 
% (modifica di "beamercolorthemebeaver.sty" in "MiKTeX 2.9/tex/latex/beamer/themes/color" )
\definecolor{darkblue}{RGB}{52,57,176}

\setbeamercolor{section in toc}{fg=black,bg=white}
\setbeamercolor{alerted text}{fg=darkblue!80!gray}
\setbeamercolor*{palette primary}{fg=darkblue!60!black,bg=gray!30!white}
\setbeamercolor*{palette secondary}{fg=darkblue!70!black,bg=gray!15!white}
\setbeamercolor*{palette tertiary}{bg=darkblue!80!black,fg=gray!10!white}
\setbeamercolor*{palette quaternary}{fg=darkblue,bg=gray!5!white}

\setbeamercolor*{sidebar}{fg=darkblue,bg=gray!15!white}

\setbeamercolor*{palette sidebar primary}{fg=darkblue!10!black}
\setbeamercolor*{palette sidebar secondary}{fg=white}
\setbeamercolor*{palette sidebar tertiary}{fg=darkblue!50!black}
\setbeamercolor*{palette sidebar quaternary}{fg=gray!10!white}

%\setbeamercolor*{titlelike}{parent=palette primary}
\setbeamercolor{titlelike}{parent=palette primary,fg=darkblue}
\setbeamercolor{frametitle}{bg=gray!10!white}
\setbeamercolor{frametitle right}{bg=gray!60!white}

\setbeamercolor*{separation line}{}
\setbeamercolor*{fine separation line}{}


%%     Impostazione dei blocchi - INNERTHEME 
% (modifica di "beamerinnerthemerounded.sty" in "MiKTeX 2.9/tex/latex/beamer/themes/inner" )
%\setbeamertemplate{blocks}[rounded]
\setbeamertemplate{items}[ball]
%\setbeamertemplate{sections/subsections in toc}[ball]
%\setbeamertemplate{title page}[default][colsep=-4bp,rounded=true]
%\setbeamertemplate{part page}[default][colsep=-4bp,rounded=true]


%	%    Impostazione delle intestazioni - OUTERTHEME
% (modifica di "beamerouterthemeinfolines.sty" in "MiKTeX 2.9/tex/latex/beamer/themes/outer" )
\setbeamercolor{institute in head/foot}{parent=palette secondary}
\setbeamercolor{subject in head/foot}{parent=palette tertiary}
\setbeamercolor{title in head/foot}{parent=palette primary}
\setbeamertemplate{footline}
{ 
	\leavevmode%
	\hbox{%
  		% "ht" è lo spazio sopra il carattere partendo dalla base del carattere, "dp"  è lo spazio sotto il carattere partendo sempre dalla base del carattere
  		\begin{beamercolorbox}[wd=.333333\paperwidth,ht=2.5ex,dp=1ex,left,leftskip=4ex]{author in head/foot}%
   			\usebeamerfont{author in head/foot}\insertshortauthor
  		\end{beamercolorbox}%
  		\begin{beamercolorbox}[wd=.333333\paperwidth,ht=2.5ex,dp=1ex,center]{institute in head/foot}%
    		\usebeamerfont{title in head/foot}\insertshortinstitute
 		 \end{beamercolorbox}%
 		 \begin{beamercolorbox}[wd=.333333\paperwidth,ht=2.5ex,dp=1ex,right,rightskip=4ex]{date in head/foot}%
 		 	\usebeamerfont{date in head/foot}\insertshortdate{}\hspace*{2em}\insertframenumber{} / \inserttotalframenumber
 		 \end{beamercolorbox}
 		 }%
 	\vskip0pt%
}
\makeatletter
\setbeamertemplate{headline}
{
  \leavevmode%
  \hbox{%
  \begin{beamercolorbox}[wd=.5\paperwidth,ht=2.8ex,dp=1.35ex,left]{subject in head/foot}	
    \usebeamerfont{section in head/foot}\hspace*{4ex}Complements of Condensed Matter Physics			% --> N.B. Qui c'è l'unica scritta che non usa nessun template tipo \title o \institute
  \end{beamercolorbox}%
  \begin{beamercolorbox}[wd=.5\paperwidth,ht=2.8ex,dp=1.35ex,right]{title in head/foot}%
    \usebeamerfont{subsection in head/foot}\insertshorttitle\hspace*{4ex}
  \end{beamercolorbox}}%
  \vskip0pt%
}
\makeatother

%%    FINE IMPOSTAZIONI TEMA
%% ---------------------------------------------------------------------------------------------------------



% settagio dati principali del file - titolo, autore, ...
\title[Selected concepts of Molecular Dynamics Simulation]{Molecular Dynamics Simulation}
\subtitle{A brief introduction}
\author{Daniele Di Bari}
\institute{University of Perugia}
\date{5th March 2018}
\subject{Complements of Condensed Matter Physics}

% By default the beamer class adds navigation buttons in the bottom right corner. To remove them one can place
\beamertemplatenavigationsymbolsempty

% settaggio prima slide
\setbeamertemplate{title page}
{
	\shadowcolor{white!30!black}
	\vspace{-0.5cm}	
	
    \shadowtext{\textcolor{white}{\textbf{\footnotesize{Selected concepts of}}}}
   
    \vspace{0.2cm}  	
    
   	\shadowtext{\textcolor{white}{\textbf{\LARGE \inserttitle}}}\par
      
   	\vspace{0.1cm}  

   	\shadowtext{\textcolor{white}{\emph{\large \insertsubtitle}}}\par
   	
   	\vspace{2.5cm}
	
	\titlepagetext{lightgray}{The general particle physics}\\
	\titlepagetext{lightgray}{experiment in space, on board}\\
	\titlepagetext{lightgray}{the International Space Station}\\
	\titlepagetext{lightgray}{since 19th May 2011.}		
}
				
% colors
\definecolor{itemred}{RGB}{163,0,0}
\definecolor{itemblue}{RGB}{52,57,176}
\definecolor{notegrey}{RGB}{90,90,90}
\definecolor{titlebgd_grey}{RGB}{242,242,242}
\definecolor{titleframe_red}{RGB}{204,0,0}

% COMANDI
% Formato testo generico in titolpage
\newcommand\titlepagetext[2]{\textcolor{#1}{\footnotesize{\textit{{#2}}}}}
\newcommand\colortextbf[2]{\textcolor{#1}{\textbf{#2}}}
\newcommand\bluetextbf[1]{\textcolor{itemblue}{\textbf{#1}}}
\newcommand\bluetextit[1]{\textcolor{itemblue}{\textit{#1}}}
\newcommand\textitem[2]{\textcolor{itemblue}{\textbf{#1} (\textit{#2})}}

% Testo giustificato
\newenvironment<>{justify}{\justifying{}}

\begin{document}
	% impostare una foto come sfondo del frame: 
	%\usebackgroundtemplate{\includegraphics[height=\paperheight,keepaspectratio]{on_ISS-08_cutMIDDLEUP-3088.jpg}}	
	% 	possibili impostazioni per l'img: width=\paperwidth  --  height=\paperheight  --  keepaspectratio

\begin{frame}\pdfbookmark[2]{Title}{Title}
	\titlepage
\end{frame}

\note{ 
\footnotesize{
There are a range of techniques of a quasi-experimental character, referred to collectively as computer simulation, the importance of which in the development of liquid state theory can hardly be overstated.
The usefulness of computer simulation relies on the fact that a sample containing a few hundred or few thousand particles is in many cases sufficiently large to simulate the behaviour of a macroscopic system. 
There are two classic approaches: molecular dynamics and Monte Carlo. 
Molecular dynamics has the advantage of allowing the study of time-dependent processes, but for the calculation of static properties a Monte Carlo method may be more efficient. 
We will focus on the first.
}
}

\usebackgroundtemplate{}

%\begin{frame}[allowframebreaks]\pdfbookmark[2]{Contents}{Contents}
%\frametitle{Table of Contents}
%\tableofcontents
%\end{frame}

\begin{frame}<presentation:0>[noframenumbering]
frame rimosso dalla presentazione
\end{frame}

\section{Introduction}
\begin{frame}
	\setbeamertemplate{background}{white}
	\frametitle{Introduction}
	\framesubtitle{The Alpha Magnetic Spectrometer}
	\justifying
	AMS-02 is a general purpose \textbf{high energy particle detector}  designed to operate as an external module on the
	International Space Station.
	
	\vspace{0.25cm}
	\begin{block}{Purpose of the AMS experiment}
		\justifying
		To perform accurate, high statistics, long duration measurements of the spectra of energetic primary charged 
		cosmic rays (CRs) in space. 
	\end{block}
		
	\vspace{0.25cm}
	Some of the \textbf{physics goals} are:
	\begin{enumerate}
		\item \bluetextbf{Dark Matter}
		\item \bluetextbf{Matter/Antimatter Asymmetry}
		\item \bluetextbf{Cosmic Ray Physics}
	\end{enumerate}
\end{frame}
\note{
	AMS-02 (Alpha Magnetic Spectrometer ) is a general purpose high energy particle detector which was successfully deployed on 
	the International Space Station (ISS) on May 19, 2011 to conduct a unique long duration mission of fundamental physics 
	research in space. Among the physics objectives of AMS are the searches for an understanding of Dark Matter, Anti-matter, the 
	origin of cosmic rays and the exploration of new physics phenomena not possible to study with ground based experiments
	Lo scopo dell'esperimento è quello di effettuare, per un periodo di lunga durata così da ottenere un'elevata statistica, delle 
	misure molto accurate dello spettro di energia (per energie fino all'ordine dei TeV) dei raggi cosmici carichi primari direttamente 
	nello spazio.
}

\section{Technical Requirements}
\subsection{from Physics Goals}
\begin{frame}
    \frametitle{Technical Requirements}
    \framesubtitle{from Physics Goals}
    \justifying
    
	Physical goals involve \textbf{technical requirements for the AMS detector}:
	
	\begin{itemize}\justifying
		\item	\bluetextbf{Dark Matter $\Rightarrow$}
					to get $e^+/p$ rejection of $\sim10^{-6}$ for the measurement of the positron fraction.
		\item	\bluetextbf{Matter/Antimatter Asymmetry $\Rightarrow$}
					to reach a sensitivity in the search for anti-matter nuclei of $10^{-10}$ (ratio of anti-helium nuclei to helium nuclei).
		\item	\bluetextbf{Cosmic Ray Physics $\Rightarrow$}
					to measure the composition and spectra of charged particles with an accuracy of 1\%.
	\end{itemize}
	
	Moreover, \bluetextbf{for each of these ones}, it is very important to extend, as far as possible, the energy range of the 
	measurements.
	
	For AMS, the required energy range is from 0.5 to $\sim$ 2000 $GeV$.

	%\footnotesize{\underline{\textbf{Note}} These requirements represent a considerable sensitivity improvement compared to the previous space-borne experiments.}

\end{frame}

\note{
	Esiste un forte interesse nell'effettuare delle misure di precisione della frazione di positroni dei raggi cosmici nella regione 
	energetica da 10 a 1000 GeV, in quanto le misurazioni di $e^+ / (e^+ + e^-)$, da AMS-01, HEAT, PAMELA e FERMI indicano 
	una grande deviazione di questo rapporto dalla produzione di e+ e e- previsto da un modello che comprende solo collisioni 
	ordinarie del raggio cosmico.
   	 
   	These available measurements are both at too low an energy and of too limited statistics to shed the light on the origin of this 
   	significant excess.
   	 
	AMS-02 is expected to provide definitive answers concerning the nature of this deviation.
}

\begin{frame}
	\frametitle{Technical Requirements}
    \framesubtitle{from Physics Goals}	
    
    \begin{block}{The technical challenge of AMS-02}
		\justifying
		The illustrated requirements represent a considerable sensitivity improvement compared to the previous space-borne 
		experiments.
	\end{block}
   	
   	\vspace{0.25cm}
	\small{\bluetextbf{NOTE}}
	\vspace{-0.2cm}
    \begin{itemize}
    	\item[\footnotesize{$\blacktriangleright$}] 
    		\footnotesize{\justifying
    		There is a strong demand for precision measurements of CRs in the energy region from 10 to 1000 
    		$GeV$ as the measurements of positron fraction, i.e.  $e^+ / (e^+ + e^-)$, by AMS-01, HEAT, PAMELA and FERMI 
    		indicate a large deviation of this ratio from the production of $e^+$ and $e^-$ predicted by a model that includes only 
    		ordinary CR collisions.
    		
    		These available measurements are both at too low an energy and of too limited statistics to shed the light on the origin of 
    		this significant excess.
    		
    		AMS-02 is expected to provide definitive answers concerning the nature of this deviation.}
   	\end{itemize}  
   	 
%
%%	\vspace{0.25cm}
%	There is a strong demand for precision measurements of cosmic rays in the energy region from 10 to 1000 GeV as the 
%	measurements of e+/(e+ + e-) by AMS-01, HEAT, PAMELA and FERMI indicate a large deviation of this ratio from the 
%	production of e+ and e- predicted by a model that includes only ordinary cosmic ray collisions. 
%	
%	These available measurements are both at too low an energy and of too limited statistics to shed the light on the origin of this 
%	significant excess. 
%	
%	AMS-02 is	expected to provide definitive answers concerning the nature of this deviation.
\end{frame}

%\begin{frame}
%	\frametitle{Technical Requirements}
%    \framesubtitle{From Physics Goals}
%	\begin{columns}
%		\begin{column}{0.5\textwidth}
%			\begin{center}
%				\includegraphics[width=0.95\columnwidth]{Schem_Requirements-DM.png}
%			\end{center}
%		\end{column}
%
%		\begin{column}{0.5\textwidth}
%			\begin{center}
%				\includegraphics[width=0.95\columnwidth]{Schem_Requirements-Antimatter.png}
%			\end{center}
%		\end{column}
%	\end{columns}
%\end{frame}

\begin{frame}[label=AMS-apparatus_img]
	\frametitle{Technical Requirement}
	\framesubtitle{general structure of the AMS apparatus}
	\begin{center}
		\vspace{-0.15cm}
		%\hspace{-0.35cm}\includegraphics[height=0.7\paperheight]{Schem_structure_of_AMS.png}

		\vspace{-0.35cm}

		\hspace{0.25cm}\hyperlink{AMS-apparatus_list}{\textbf{\beamerbutton{link to AMS apparatus list}}}
	\end{center}
	%\vspace{-0.25cm}
\end{frame}

\end{document}
